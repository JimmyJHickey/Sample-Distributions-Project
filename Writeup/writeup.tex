\documentclass[12pt]{article}

% Margins
\usepackage[letterpaper, top=1in, bottom=1in, left=1in, right=1in]{geometry}

% Text over arrows
\usepackage{mathtools}

% Graphics and subcaptions
\usepackage{graphicx}
\usepackage{float}

% For code
\usepackage{courier}
\usepackage{listings}
\lstset{ mathescape }
\lstset{basicstyle=\ttfamily\footnotesize,breaklines=true}

% Remove Default sections headers
\setcounter{secnumdepth}{0}

% Clickable Table of Contents
\usepackage{color}   %May be necessary if you want to color links
\usepackage{hyperref}
\hypersetup{
	colorlinks=true, %set true if you want colored links
	linktoc=all,     %set to all if you want both sections and subsections linked
	linkcolor=black,  %choose some color if you want links to stand out
}

% Change Font
\usepackage[sfdefault]{roboto}  %% Option 'sfdefault' only if the base font of the document is to be sans serif
\usepackage[T1]{fontenc}


\usepackage[english]{babel}
\usepackage[utf8]{inputenc}
\usepackage{fancyhdr}

% For double spacing
\usepackage{setspace}

\usepackage[english]{babel}
\usepackage[utf8]{inputenc}
\usepackage{fancyhdr}

\pagenumbering{arabic}

\pagestyle{fancy}
\rhead{Hickey, Kovach, Saunders, Stewart}

\lhead{ST 501 R Project}

% Add \tab command
\newcommand\tab[1][1cm]{\hspace*{#1}}

% Add \mean bar command
\newcommand*\mean[1]{\bar{#1}}

%opening
\title{ST 501 R Project}
\author{Jimmy Hickey, Shaleni Kovach, Meredith Saunders, Stephanie Stewart}



\begin{document}
\maketitle
\tableofcontents
\clearpage

\doublespacing

\section{Part I - Convergence in Probability}

\subsection{1.}
Consider the "double exponential" or Laplace Distribution. A RV $Y \sim Laplace(\mu, b)$ has the PDF given by

$$f_Y(y) = \frac{1}{2b}e^{-\Big(  \frac{|y-\mu |}{b}   \Big) } $$

for $-\infty < y < \infty$, $-\infty < \mu < \infty$, and $b>0.$

We will consider having a random sample of Laplace RVs with $\mu = 0$ and $b = 5$. We'll look at the limiting behavior of $L=\frac{1}{n}\sum_{i=1}^{n}Y_i^2$ using simulation.

\subsubsection{a.}
Give a derivation of what $L$ converges to in probability. You should show any moment calculations and state the theorem(s) you use.

\bigskip


By the Weak Law of Large Numbers, we know that,

$$L = \frac{1}{n}\sum_{i=1}^{n}Y_i^2 \xrightarrow{\text{p}} E(Y^2).$$

We can calculate  $E(Y^2)$ using the definition of an expected value.

\begin{align*}
	 E(Y^2) & = \int_{-\infty}^{\infty} y^2 \cdot \frac{1}{2b} \cdot e^{-\Big(  \frac{|y-\mu |}{b} \Big)} dy\\
	 & =  \int_{-\infty}^{\infty} (x + \mu)^2 \cdot \frac{1}{2b} \cdot e^{-  \frac{|x|}{b}} dx & \text{taking } x = y - \mu\\
	 & = \frac{ 1 }{2b } \int_{-\infty}^{\infty} (x^2 + 2 \mu x  + \mu^2) \cdot e^{-  \frac{|x|}{b}}dx\\
	 & = \frac{ 1 }{2b } \Big[ \int_{-\infty}^{\infty} x^2 \cdot e^{-  \frac{|x|}{b}}dx+ \int_{-\infty}^{\infty} 2\mu x \cdot e^{-  \frac{|x|}{b}}dx+ \int_{-\infty}^{\infty} \mu^2 e^{-  \frac{|x|}{b}}dx \Big]\\
	 & = \frac{ 1 }{ 2b }[4b^3 + 0 + 2b\mu^2]\\
	 & = \frac{ 4b^3 }{ 2b } + \frac{ 2b\mu^2 }{ 2b }\\
	 & = 2b^2 + \mu
\end{align*}

We can confirm this by checking $ E(Y)^2 = Var(Y) + E(Y)^2$. From Wikipedia, we can see that $E(Y) = \mu$ and $Var(Y) = 2b^2$.

$$Var(Y) + E(Y)^2 = 2b^2 + (\mu)^2 = 2b^2 + \mu^2 = E(Y)^2$$

In the case of $\mu =0, \ b = 5$, we get that $L \xrightarrow{\text{p}} 2\cdot 5^2 + 0 = 50$.

\subsubsection{b.}
Explain what $K = \sqrt{L}$ converges to and why.

\bigskip

By the Continuity Theorem, we can see that $K = \sqrt{L}  \xrightarrow{\text{p}}  \sqrt{2b^2 +\mu^2}$. In the case of $\mu =0, \ b = 5$, we get that $K \xrightarrow{\text{p}}  \sqrt{50}$.

\subsubsection{c.}
Derive the CDF of $Y$ . Note you’ll have two cases and you should show your work.

\bigskip
Our CDF looks like
$$F_Y(y) = \int_{-\infty}^{y}\frac{1}{2b}e^{-\Big(  \frac{|x-\mu |}{b}   \Big) }  dx$$

Using the absolute value, we can split the density function into two pieces, $y< \mu$ and $y\geq \mu$. Let us examine the first case.

\begin{align*}
F_Y(y) & = \int_{\infty}^{y} \frac{1}{2b}e^{-\frac{\mu-x}{b} } dx & \text{for } y <\mu\\
	& =  \int_{\infty}^{y} \frac{1}{2b}e^{\frac{x-\mu}{b} } dx \\
	& = \frac{1}{2} e^{\frac{y-\mu }{b}}
\end{align*}

Next we can examine the $y\geq \mu$ case.

\begin{align*}
F_Y(y) & = \int_{\infty}^{y} \frac{1}{2b}e^{-\frac{x - \mu}{b} } dx & \text{for } y \geq\mu\\
& =  \int_{\infty}^{\mu} \frac{1}{2b}e^{\frac{\mu - x}{b} } dx + \int_{\mu}^{y} \frac{1}{2b}e^{\frac{\mu - x}{b} } dx  \\
& = \frac{ 1}{ 2 } + \Big(\frac{ 1}{ 2 } - \frac{ 1 }{ 2 } e^{\frac{ \mu - y }{ b }}\Big) \\
& = 1 - \frac{ 1 }{ 2 } e^{\frac{ \mu - y }{ b }}
\end{align*}

Putting the pieces together gives,

$$F_Y(y) = \begin{cases}
 \frac{1}{2} e^{\frac{y-\mu }{b}} & \text{for } y <\mu \\
1 - \frac{ 1 }{ 2 } e^{\frac{ \mu - y }{ b }} &  \text{for } y \geq\mu.
\end{cases}$$

\subsubsection{d. \& e.}

The code for parts d. and e. can be found in the \lstinline{Problem_1.R} file. Here are the resulting graphs.



\begin{figure}[H]
	\centering
	\includegraphics{img/n_vs_Ln.png}
	\caption{$L$ as $n$ increases}
\end{figure}

\begin{figure}[H]
	\centering
	\includegraphics{img/n_vs_Kn.png}
	\caption{$K$ as $n$ increases}
\end{figure}

These graphs both demonstrate that our RVs L and K are converging in probability At each sample size n, the same number of samples (50) were taken.  Notice the trend as the number of observations (n) in each sample increases. It is clear that there is far less spread. As n increases, the RVs are converging to the blue line. The observed values of L are geting closer and closer to 50. And the observed values of K are approaching sqrt(50).  As n increases we could continue to shrink our epsilon bubble around the expected value and we will continue to see this convergence.


\section{Part II - Convergence in Distribution}

\subsection{2.}
This time we’ll look at the limiting distribution of our statistics.

\subsection{3.}
Theoretically, since $L$ is an average of \textit{iid} RVs ($Y_i^2$ß are each RVs) with finite variance
we know that, properly standardized, $L$ should have a standard normal limiting distribution by the CLT. 

Derive the appropriate standardization that will converge to a standard normal distribution for $\mu = 0$ and an arbitrary $b$. Show your work. Note: the kurtosis of the
Laplace distribution is 6 and we have $\mu = 0$. Find the formula for kurtosis (book or
wikipedia) and the calculation of the fourth moment won’t be too bad!

\subsection{4.}
Redo your above 4 plots using the standardization.

\subsection{5. \& 6.}
 Now let’s see if convergence is occurring for larger values of $n$. Generate data using the
same method as above except do so for $n = 1000$ and $n = 10,000$. Use $N = 10,000$
data sets as well.
Create similar plots for these two n values. In a comment discuss how the CLT is
manifesting for this problem. Does n > 30 work?

Using $N=50$, while the samples seem to be converging to normal, they still do not appear quite normal even for sample size $n=10,000$.
When we change to $N=10,000$ both $n=1,000$ and $n=10,000$ appear normal.
This shows it is not enough for $n>30$ the number of sample repetitions also plays a role in CLT..

\begin{figure}[H]
	\centering
	\includegraphics{img/Problem6_1.png}
	\caption{50 samples at $n = 1,000$}
\end{figure}

\begin{figure}[H]
	\centering
	\includegraphics{img/Problem6_2.png}
	\caption{50 samples at $n=10,000$}
\end{figure}

\begin{figure}[H]
	\centering
	\includegraphics{img/Problem6_3.png}
	\caption{10,000 samples at $n=1,000$}
\end{figure}

\begin{figure}[H]
	\centering
	\includegraphics{img/Problem6_4.png}
	\caption{10,000 samples at $n=10,000$}
\end{figure}



\end{document}
